\documentclass[11pt]{article}
\usepackage{fullpage}
\usepackage[utf8]{inputenc}
\usepackage{listings}
\usepackage{amssymb}


\lstset{basicstyle=\footnotesize, numbers=left, numberstyle=\footnotesize,
  stepnumber=1, numbersep=5pt, captionpos=b, frame=single,
  showstringspaces=false}

\begin{document}

DistAlgo is a high-level language for writing distributed algorithms, such that
they serve as both clear specifications and runnable implementations of those
algorithms. This document provides a brief, practical introduction to the major
features of the language using several iterations of a familiar example.

DistAlgo is currently implemented as an extension of the Python language, and
requires Python 3.4 or higher. This tutorial assumes some familiarity with
Object-Oriented programming in general, and with Python, in particular.

\section{Main}

Let's begin with the simplest possible version of ``Hello World'' written in
DistAlgo.

\lstset{language={python}, morekeywords={send, sent, received, each, some,
    await, new, setup, start, run, process}}

\begin{lstlisting}[caption={hello\_world01.da - Main Function Definition}, label={lst:hw01}]
    def main():
        print('Hello World.')
\end{lstlisting}

\noindent Every DistAlgo program must have a main function. In this case, the body of
\texttt{main} merely contains a call to Python's print function, to output
``Hello World'' to standard output. As we will see in the next example, the
inteded use of the main function in a DistAlgo program is to create, prepare,
and then begin the execution of the distinct processes that participate in the
distributed algorithm.

\section{Processes}

In the second version of ``Hello World'' we can see how process definition and
creation works in a DistAlgo program.

\begin{lstlisting}[caption={hello\_world02.da - Process Definition}, label={lst:hw02}]
  class P (process):
      def setup(name):
          pass

      def run():
          output('Hello World from:', self.name, ', AKA:', self.id)

  def main():
      ps = new(P)
      setup(ps, ('bob',))
      start(ps)
\end{lstlisting}

\noindent Distributed algorithms are constituted by the interaction of multiple, distinct
processes. DistAlgo is intended for the implementation of distributed
algorithms. To that end, DistAlgo makes it easy to define new process types
that will execute the behavior required by the algorithm.

Every user defined process type in DistAlgo is an extension of the base class
\texttt{process}. In the second version of ``Hello World'' we define a class
called \texttt{P}, which extends \texttt{process}. In order to properly define
a new type of DistAlgo process, the user must define two functions:
\texttt{setup} and \texttt{run}.

The \texttt{setup} method is used to declare and initialize any instance
variables of the user's new process class. Any parameters of the setup method
are implicitly declared as instance variables and initialized with the value
passed as the argument to the parameter. In the second example, \texttt{setup}
has one parameter \texttt{name}, which is initialized with the value
\texttt{'bob'}. In addition, every instance of a distalgo process has a unique
identifier stored in an instance variable, \texttt{self.id}, which is the pair
of the IP address and the port number upon which the process instance is
listening.

The \texttt{run} method of a DistAlgo process contains the definition of the
behavior of the process. For the example in listing 2, the process merely
prints a ``Hello World'' message, identifying itself using both the name we
supplied as an argument and its unique id.

In this second example we can also see a proper use of the \texttt{main} method
of a DistAlgo program. Here we have calls to three Distalgo methods:
\texttt{new}, \texttt{setup}, and \texttt{start}.

The \texttt{new} method is used to instantiate a set of DistAlgo process
instances. We supply the classname of the process class we wish to instantiate,
\texttt{P}, as an argument, and \texttt{new} returns a set of references to the
new process instances. In this case we store the set in the variable
\texttt{ps}.  If the class name is the only argument, then \texttt{new} creates
just one instance of that process class. However, even if \texttt{new} is used
to create just once process instance, the returned value is still a set--in
this case a singleton set--containing a reference to that process instance.

The \texttt{setup} method is used to initialize our new process instances. The
first argument is a set of process instance references. The second argument is
a tuple of arguments to be passed to the parameters of the \texttt{setup}
method defined for the process class corresponding to the class of the process
instances one is trying to initialize. Note, even if there is only one argument
passed to the \texttt{setup} method of the process class, it must still be
packaged within a singleton tuple.

Finally, we can begin the execution of the process instances by passing a set
of references to those process instances to the \texttt{start} method. The call
to \texttt{start} will cause the \texttt{run} method defined within the
corresponding process class to begin executing at each process instance
included in the call to \texttt{start}.

In the third example, we can see how multiple process instances are created.

\begin{lstlisting}[caption={hello\_world03.da - Multiple Process Instances}, label={lst:hw03}]
  class P (process):
      def setup(name):
          pass

      def run():
          output('Hello World from:', name, ', AKA:', self.id)

  def main():
      num_processes = 2
      ps = new(P, num = num_processes)
      setup(ps, ('bob', ))
      start(ps)

\end{lstlisting}

\noindent You can run this program from the command line by entering:
``python3 -m da hello\_world3.da''.

This program is very similar to ``hello\_world2.da''. In fact, the definition of
the process class, \texttt{P}, is identical to the one given in the previous
example. There are just two changes in \texttt{main}. First, we add a new
variable, \texttt{num\_processes}, which is initialized to $2$. Second, we pass
this new variable as the argument to the \texttt{num} parameter of the
\texttt{new} method, resulting in the instantiation of two process instances of
the process class, \texttt{P}. These two instances are both initialized by the
call to \texttt{setup}, and then set running by the call to \texttt{start}.

The instantiation and execution of the two process instances can be observed in
the output when the program is run. There will be two ``Hello World'' outputs,
each with a distinct value for \texttt{self.id}, though both will claim that
their name is 'bob', since they are both inititialized with the same parameter.
You should think about how you might initialize the \texttt{name} fields of
these two process instances with distinct values.

This latest version of hello\_world features two DistAlgo process instances,
but they do not interact with or even refer to each other. Distributed
algorithms are intended to feature this sort of interplay between distinct
processes, so let's add some in the next example.

Here is the fourth example:

\begin{lstlisting}[caption={hello\_world04.da - Multiple Processes Referring to Each Other}, label={lst:hw04}]
class P (process):
    def setup(processes):
        pass

    def run():
        for p in processes:
            output(self.id, 'says Hello to: ', p)

def main():
    num_processes = 2
    ps = new(P, num = num_processes)
    setup(ps, (ps,))
    start(ps)  
\end{lstlisting}

\noindent You can run this program from the command line by entering:
``python3 -m da hello\_world04.da''. There are several changes to the program.
First, we have altered the definition of the \texttt{setup} method so that it
accepts parameter, \texttt{processes}, which will hold a set of references to
DistAlgo process instances. Next, we have altered the the \texttt{run} method
of \texttt{P} so that instead of greeting the entire world, it instead loops
over all the process instances in \texttt{processes} and sends each of them
an individualized greeting, <process id> says Hello to <process id>. 

Within the main method we generate two distinct process instances of type
\texttt{P}, and store the set of their references in texttt{ps}. We then
initialize both instances passing that same set of references to \texttt{setup}
to serve as the argument to texttt{processes}. Finally, we start those process
instances running as before.

If you run the program then you will see that in the ouput there are four
``X say Hello to Y''. If we label the two process instances, $p1$ and $p2$,
then the four greetings have the following forms: i) ``p1 says Hello to p1'';
ii) ``p1 says Hello to p2''; iii) ``p2 says Hello to p1'';
iv) ``p2 says Hello to p2'', in some arbitrary order. But, we probably did not
intend to have the processes greeting themselves. So what went wrong?

The problem occurs on line 12. We passed the entire set ps to the
\texttt{processes} parameter of \texttt{P} for each process instance, but there
is a reference to that process instance in the set \texttt{ps}. So, each ends
up with a \texttt{processes} set that contains a reference to itself. How can
we avoid this?

One way to address this problem is to prevent a process instance from
contacting itself while it executes its \texttt{run} method. We can do this
with a conditional, as illustrated in Example 5:

\begin{lstlisting}[caption={hello\_world05.da - Excluding Self-References, the Ugly Way}, label={lst:hw05}]
  class P (process):
      def setup(processes):
          pass

      def run():
          for p in processes:
              if p != self.id:
                  output(self.id, 'says Hello to: ', p)

  def main():
      num_processes = 2
      ps = new(P, num = num_processes)
      setup(ps, (ps,))
      start(ps)
\end{lstlisting}

\noindent The only change made from Example 4 is the insertion of an \texttt{if}
statement within the loop of the \texttt{run} method that sends the greetings
to the processes in the set \texttt{processes}. This \texttt{if} only allows
the process instance to execute the \texttt{output} statement if the process id
contained in the loop control variable, \texttt{p}, is distinct from the
process id of the process instance executing the code (\texttt{self.id}).

Adding this conditional statement solves our problem, but it does so in a crude
and inefficient manner. Real distriubted algorithms will involve more
complicated and more numerous interactions between process instances. In order
to ensure that a process never attempted to unecessarily interact with itself
each interaction statement in the program would have to be guarded by a
conditional like this one. That is many extra lines of code that need to be
carefully checked to guarantee you have not damaged the control logic of the
program. We need a better solution than this.

It turns out there is a common pattern present in many DistAlgo examples that
allows one to initialize a process with a set of references to other process
instances, but exclude a reference to the process instance one is attempting to
initialize. It is illustrated in Example 6:

\begin{lstlisting}[caption={hello\_world06.da - Excluding Self-References, the Nice Way}, label={lst:hw06}]
class P (process):
    def setup(processes):
        pass

    def run():
        for p in processes:
        output(self.id, 'says Hello to: ', p)

def main():
    num_processes = 2
    ps = new(P, num = num_processes)
    for p in ps:
        setup(p, (ps - {p}, ))
    start(ps)
\end{lstlisting}

The only difference between Example 6 and Example 4 occurs inside the call to
the \texttt{setup} method within main. We initialize each of the process
instances one at a time within a loop. For each process instance, instead of
passing the entire set \texttt{ps} to the \texttt{processes} paramter of
\texttt{P}, we instead pass the set produced taking the difference of
\texttt{ps} with a singlton set that contains a reference to the process
instance being initialized. Thus, each process instance is set up with a set of
refereces to every other process instance in \texttt{ps}.

If you run Example 6 by entering ``python3 -m da hello\_world06.da'', then you
will see that there now only two outputs. $p1$ greets $p2$, and $p2$ greets
$p1$. All is once again right with the world. Each process instance greets the
other without also greeting itself.

There are many examples of DistAlgo programs in which set difference is used in
this way to create a set of processes that includes all the processes in a set
to which the current process belongs, but which excludes the current process.
This is not surprising. Process classes define instances that will interact
with each other, but they do not usually need to interact with themselves. It
would be ugly to check the identity of the process the current instance is
attempting to interact with to verify that it is not unnecessarily
communicating with itself. Using set difference in this way lets us avoid all
that ugliness.

\section{Process Interaction}

So far we have seen how to define the process classes that will play the roles
in the distributed algorithm we are attempting to specify. However, distributed
algorithms require these roles to interact by sending and receiving messages
among each other. In this section we will introduce the DistAlgo constructs
that enable process instances to send messages to and receive messages from
each other.

Here is Example 7:
\begin{lstlisting}[caption={hello\_world07.da - Send}, label={lst:hw07}]
  class P (process):
      def setup(processes):
          pass

      def run():
          for p in processes:
              send(('hello', self.id), to = p)

  def main():
      num_processes = 2
      ps = new(P, num = num_processes)
      for p in ps:
          setup(p, (ps - {p}, ))
      start(ps)
\end{lstlisting}

\noindent This program, ``hello\_world07.da'' is basically identical to the
program given in Example 6. The only difference is the statement inside the
\texttt{for} loop inside the \texttt{run} method of process \texttt{P}, where
we have replaced the \texttt{output} statement with a new kind of statement
\texttt{send}.

The \texttt{send} method is available to every class that subclasses the
DistAlgo \texttt{process} class. The \texttt{send} method take two arguments.
The first argument is the content of the message to be sent. This is typically
a tuple of various values. By convention, the first value in the tuple is a
string used as a tag for the message--it identifies the type of message being
sent. In this example, we have tagged the message in the \texttt{send} with the
value \texttt{'hello'}. This will let the recipient know that it is a greeting
message.

The second argument, prefaced by ``\texttt{to=}'', is the process id of
the process instance that is the intended recipient of the message. The sending
process instance will open a socket to the stated recipient process using its
id (recall that a process id is a tuple of the ip address and port number upon
which that process is listening for incoming connections).

If we run this program with the command ``python3 -m da hello\_world07.da'',
then you will see that there is no observable output. What has happened? The
problem is that sending messages is only half of the work we need to do in
order to make these process instances genuinely interact. We also need to
specify what a process instance should do when it \textit{receives} a message
with a \texttt{'hello'} tag.

This leads us to Example 8:
\begin{lstlisting}[caption={hello\_world08.da - Receive Handler}, label={lst:hw08}]
  class P (process):
      def setup(processes):
          pass

      def run():
          for p in processes:
              send(('hello', self.id), to = p)

      def receive(msg=('hello', pid), from_ = greeter):
          output(greeter, 'says Hello to', self.id)

  def main():
      num_processes = 2
      ps = new(P, num = num_processes)
      for p in ps:
          setup(p, (ps - {p}, ))
      start(ps)
\end{lstlisting}

\noindent The program in Example 8 is derived from the program in Example 7 by adding a
single new construct. This is the receive handler. In the Python implementation
of DistAlgo a receive handler is the definition of a method \texttt{def
  receive()}. The parameters of the \texttt{receive} method determine the
content of the message it is intended to receive and the identity of the
sender. In particular, the first parameter--the \texttt{msg=} parameter--
defines a pattern that a message must match in order to activate the body of
the receive handler. The body of the \texttt{receive} method determines the
behavior of the process instance upon receiving such a message.

In Example 8, the receive handler pattern includes the literal string tag
'hello' and an additional parameter, \texttt{pid}. Thus, this receive
handler will be activated by the receipt of any message whose first value is
the string 'hello', and which has only one other value. That second value will
be assigned to the variable \texttt{pid} so that it can be used within the body
of the receive handler. (Note: it is expected that the argument to the 
\texttt{pid} parameter will be a DistAlgo process instance id, but this will
not be enforced by Python's dynamic typing system. If the type of the parameter
matters in your program you will have to enforce it manually in the receive
handler body.)

Observe that this exactly matches the structure of the message that is being
sent in the \texttt{run} method above. This is intentional. In order for sent
messages to be handled correctly, the pattern defined for the \texttt{msg=}
parameter must be able to match the structure and tag of the content of the
sent message. If the message we sent had contained a different tag, or a
different number of additional values, then there would be no match and the
body of this receive handler would not execute.

So, let's run this version of the hello\_world program by entering: 
``python3 -m da hello\_world08.da'', and observe the results. Except, once again
nothing appears to happen. What has gone wrong? Nothing has gone wrong, but we 
need to consider more carefully how message handling in a language built for
distributed algorithms should work.

It cannot be the case that messages can be handled at any point during program 
execution. Consider the possibility that the execution of the body of the 
\texttt{run} method (or, code at other locations in the program) may depend
upon values that can be altered by the receipt of incoming messages. If 
multiple incoming messages are handled willy-nilly, then these values can be
changed before the computations that depend upon them can be fully evaluated.
Message handling may interfere and ruin the execution of our program if it is
not constrained in some way.

The solution in DistAlgo is to introduce particular locations within the code
at which incoming messages (if any) may be handled. These locations are called
yield points and they are declared within the program using labels. In order to
understand what you are doing when you include a label recognize that the code
between two labels is treated as atomic--that code will complete execution 
without any interruption by messaging events included in the distributed 
algorithm. It is important to keep this faux atomicity in mind when deciding
where to place labels in your program.

In Example 9 we add a label to our hello\_world program to handle ensure the
greeting message gets handled. Here it is:
\begin{lstlisting}[caption={hello\_world09.da - Yield Points}, label={lst:hw09}]
class P (process):
    def setup(processes):
        pass

    def run():
        for p in processes:
            send(('hello', self.id), to = p)
        -- rcv_greeting

    def receive(msg=('hello', pid), from_ = greeter):
        output(pid, 'says Hello to', self.id)
        
def main():
    num_processes = 2
    ps = new(P, num = num_processes)
    for p in ps:
        setup(p, (ps - {p}, ))
    start(ps)
\end{lstlisting}

Try running this program several times. You should notice that something is
amiss. Most of the time you will see that the process instances exchange
greeting messages. However, on some runs of the program, only one of the
greeting  messages gets output. This is because only one of the greeting
messages is handled by its recipient. Why does this happen?

Network communication can take an arbitrary amount of time to complete.  You
cannot count on messages arriving in a timely fashion in your algorithm (at
least, not without doing a lot more work). In the runs featuring the missing
output, the process instance that fails to handle the message reaches the label
\texttt{--rcv\_greeting} before the greeting message actually arrives. Since
there are no messages waiting when it reaches the label, there is nothing to
handle, and so the receive handler is not invoked. The execution falls through
past the label, and since there is no further code in the \texttt{run} method
the process instance completes execution and ends.

In order to gain more control over process interaction we need to do more than
merely create opportunities for one process instance to handle messages that
may have arrived from the other. We need to synchronize the behavior of the
two processes to create genuine interaction in this case. We will accomplish
this by using the \texttt{await} keyword. Declaring an await statment will
allow us to halt the progress of the \texttt{run} function while we wait for
some specified condition to obtain.

We can see how this is applied to the present problem by looking at Example 10:

\begin{lstlisting}[caption={hello\_world10.da - Synchronizing with Await Statements}, label={lst:hw10}]
class P (process):
    def setup(processes):
        pass

    def run():
        for p in processes:
            send(('hello', self.id), to = p)
        -- rcv_greeting
        await(False)

def receive(msg=('hello', self.id), from_ = greeter):
    output(greeter, 'says Hello to', self.id)

def main():
    num_processes = 2
    ps = new(P, num = num_processes)
    for p in ps:
        setup(p, (ps - {p}, ))
    start(ps)
\end{lstlisting}

\end{document}
