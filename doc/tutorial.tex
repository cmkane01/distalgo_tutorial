\documentclass[11pt]{article}
\usepackage{fullpage}
\usepackage[utf8]{inputenc}
\usepackage{listings}
\usepackage{amssymb}


\lstset{basicstyle=\footnotesize, numbers=left, numberstyle=\footnotesize,
  stepnumber=1, numbersep=5pt, captionpos=b, frame=single,
  showstringspaces=false}

\begin{document}

DistAlgo is a high-level language for writing distributed algorithms, such that
they serve as both clear specifications and runnable implementations of those
algorithms. This document provides a brief, practical introduction to the major
features of the language using several iterations of a familiar example.

DistAlgo is currently implemented as an extension of the Python language, and
requires Python 3.4 or higher. This tutorial assumes some familiarity with
Object-Oriented programming in general, and with Python, in particular.

Let's begin with the simplest possible version of ``Hello World'' written in
DistAlgo.

\lstset{language={python}, morekeywords={send, sent, received, each, some,
    await, new, setup, start, run, process}}

\begin{lstlisting}[caption={Hello World 01 - Main Function Definition}, label={lst:hw01}]
    def main():
        print('Hello World.')
\end{lstlisting}

Every DistAlgo program must have a main function. In this case, the body of
\texttt{main} merely contains a call to Python's print function, to output
``Hello World'' to standard output. As we will see in the next example, the
inteded use of the main function in a DistAlgo program is to create, prepare,
and then begin the execution of the distinct processes that participate in the
distributed algorithm.

In the second version of ``Hello World'' we can see how process definition and
creation works in a DistAlgo program.

\begin{lstlisting}[caption={Hello World 02 - Process Definition}, label={lst:hw02}]
  class P (process):
      def setup(name):
          pass

      def run():
          output('Hello World from: ', self.name)

  def main():
      p = new(P)
      setup(p, 'bob',))
      start(p)
\end{lstlisting}

Distributed algorithms are constituted by the interaction of multiple, distinct
processes. DistAlgo is intended for the implementation of distributed
algorithms. To that end, DistAlgo makes it easy to define new process types
that will execute the behavior required by the algorithm.

Every user defined process type in DistAlgo is an extension of the base class
\texttt{process}. In the second version of ``Hello World'' we define a class
called \texttt{P}, which extends \textt{process}. In order to properly define
a new type of DistAlgo process, the user must define two functions:
\texttt{setup} and \texttt{run}.

The \texttt{setup} method is used to declare and initialize any instance
variables of the user's new process class. Any parameters of the setup method
are implicitly declared as instance variables and initialized with the value
passed as the argument to the parameter. In the second example, \texttt{setup}
has one parameter \texttt{name}, which is initialized with the value
\texttt{'bob'}. 
\end{document}

